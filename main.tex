\documentclass[12pt]{article}

\usepackage{fullpage} % Package to use full page
\usepackage{parskip} % Package to tweak paragraph skipping
\usepackage{tikz} % Package for drawing
\usepackage{amsmath}
\DeclareMathOperator*{\argmax}{argmax}
\usepackage{hyperref}

\title{Oracle-Agent Problem}
% \author{}
\date{10/22/2018}

\begin{document}

\maketitle

\section{Setup}

Assume a very simplified contract design: all contracts are of the same size and are completely bilateral (aka each counterparty always has the same amount of margin in the contract). 
\begin{align*}
m &= \text{margin each counterparty puts in each contract}\\
N &= \text{number of contracts in system}\\
2mN &= \text{total margin in entire system}
\end{align*}

As a massive (and temporary) simplifying assumption, assume this is a \textbf{static} world---there is only one vote that occurs. During this voting period, all counterparties pay a fee or tax $\tau$ that is the service fee paid to the oracle for the service it provides (in this one-time, static setup). The sum of all these fees is $\Gamma$. 
\begin{align*}
\tau &= \text{fee or tax paid by each CP each voting period}\\
\Gamma &= \text{total fees paid by all CPs each voting period}\\
       &= 2 \tau N
\end{align*}

Assume the oracle consists of $V$ one-time, voting tokens. Each voting agent $v_i$ pays a cost $c$ to own one of the voting tokens (and in this static world the token disappears after the vote). After buying a voting token and voting, the voter receives a payoff $\varepsilon$ that should be $\geq c$ if they vote ``truthfully'' or $<c$ if they did not.
\begin{align*}
c &= \text{cost to buy voting token}\\
v_i &= \text{individual voting agent}\\
V &= \text{total number of voters}\\
\varepsilon (v_i, v^{-i}) &= \text{payoff function for $v_i$ based on actions of other voters $v^{-i}$}
\end{align*}

\subsection{Defining the Agent Problem}

Assume that each vote is binary, and there is some objective truth $S = \{0,1\}.$ Ahead of each vote, the voting agent receives a signal $s$ where $s = S$ with some (high) probability $p$. (Said differently, most voting agents should have a ``truthful'' signal):
\[ 
s = \left \{
  \begin{tabular}{cc}
  S & \text{with prob $p$}\\
  1-S & \text{with prob $1-p$}
  \end{tabular}
% \right \}
\]

Each voting agent makes a decision $x$ from a set of three options: to purchase a token for cost $c$ and vote 0, to purchase a token for cost $c$ and vote 1, or to not purchase a token and to not vote. The voting agent wants to maximize their payout:

\[
\max_{x \in \{0, 1, \emptyset \} } E [\varepsilon (x, v^{-i}) | s] - \prod_{x \in \{0,1\}} c
\]

\subsection{Defining the Oracle Problem}

The oracle wants to minimize the total amount that needs to be paid to induce the voting agents to tell the truth subject to the budget and incentive compatibility constraints:

\begin{align*}
\min \sum_i \varepsilon ( v_i, v^_{-i} ) &&\text{Oracle Problem}\\
\text{s.t.} &&\\
\sum_i \Big( E [ \varepsilon (v_i, v^{-i} ) ] - c \Big) \leq \Gamma &&\text{Budget Constraint}\\
\argmax_x  E [ \varepsilon (v_i, v^{-i} ) ] = s &&\text{Incentive Compatible}
\end{align*}

\subsection{Solving the Model}

We want to solve two problems:
\begin{enumerate}
  \item The optimal $\varepsilon$ payout function using the Mirrlees approach.
  \item The optimal $\tau$ for the fee/tax using the Ramsey approach. Note that this $\tau$ only applies to this static, one-time world.
\end{enumerate}

\subsection{Next Steps}

Based on the learnings from solving the static world exercise, we (hopefully) can use the same functional form for the ideal (Mirrlees) payout function to extend this model to the dynamic world and solve for the simpler Ramsey problem of the optimal flat tax rate for the dynamic system.
% \bibliographystyle{plain}
% \bibliography{bibliography.bib}
\end{document}