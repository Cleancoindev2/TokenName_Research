\documentclass[12pt]{article}

  % Get right packages
  \usepackage{amsmath}
  \usepackage{amssymb}
  \usepackage{amsthm}
  \usepackage{fullpage} % Package to use full page
  \usepackage{hyperref}
  \usepackage{parskip} % Package to tweak paragraph skipping

  % User commands
  \newtheorem{thm}{Theorem}
  \DeclareMathOperator*{\argmax}{argmax}

  % Title info
  \title{Oracle-Agent Problem}
  \author{}
  \date{10/22/2018}

\begin{document}

\maketitle


\section{Introduction}

  The UMA (Universal Markets Access) protocol is meant to provide universal access to financial
  markets so that everyone might have the opportunity to benefit from sharing risks with others. The
  way that we will achieve this is by building a platform for trading this risk and an
  \textit{oracle} that will dictate what market prices are for various forms of risk... In order to
  support having a non-trivial amount of risk on the UMA protocol, we need to ensure that the oracle
  has the ability to produce accurate price information, and, in particular, it must be resistant to
  bribery attacks where an individual attempts to move prices in their favor through corrupt means.

  This document describes a model which we use to influence decisions about how to ensure that the
  cost of corruption (CoC) is higher than the profit from corruption (PfC). The system is
  economically stable as long as the $\text{CoC} > \text{PfC}$. All things equal, it is best if the
  system generates a large spread between CoC and PfC to ensure that attacking the system requires
  as high a cost as reasonable, however, raising this spread will likely come at the cost of the
  efficiency of the system so it is important to think about the externalties imposed by the oracle
  on the trading protocol.

  We begin by describing a static environment that allows us to think about the tradeoffs faced when
  attempting to raise the CoC-PfC spread and ensuring that the system is difficult to corrupt.


\section{Static Environment}

  In the static world, the trading protocol generates $T = 2 \tau N$ revenue and has $mN$ margin
  that is exposed to being seized if the Oracle reports incorrect information\footnote{For the
  assumptions that generate these outcomes, see Appendix A}.

  There is a random state of the world $S \in \{0, 1\}$. There are three types of agents:

  \begin{enumerate}
    \item The oracle: Responsible for reporting $\hat{S}$. The oracle wants to choose $\hat{S} = S$
    but cannot observe $S$.
    \item The malevolent: Can observe $S$ and receives $mN$ if the oracle reports $\hat{S} = 1 - S$.
    \item The individuals: Can observe $S$ and are incentivized by the oracle and the malevolent
    into voting $x = \{0, 1\}$.
  \end{enumerate}

  There are $K$ agents who decide whether to truthfully report the state to the Oracle\dots A truthful
  report entails $x = S$ and an untruthful report is $x = 1 - S$. Agents vary in their aversion to
  lying which is measured by $l \sim F$.

  The Oracle's goal is to report the true state of world $\hat{S} = S$ in spite of the fact that it
  is, to the Oracle, unobservable. The Oracle does this by offering to pay the agents who vote
  $\gamma(x, X)$ where $x$ is an individual agent's vote and $X$ is the number of agents who vote
  $x=0$\footnote{ We could have also chosen $x=1$; as soon as we know how many people vote 0 (1)
  then we know how many people voted 1 (0).}. After receiving the votes made by the agents, the
  Oracle reports whichever option received the majority vote. The payments to the voters are
  denominated in \textit{rights to vote for tomorrow} and are also claims to a percentage of the tax
  revenue. The \textit{rights to vote for tomorrow} are valued at
  $p' = \sum_{t=1}^\infty \left(\frac{1}{1+r} \right)^t \frac{T}{K}$
  if the Oracle reports the truth and 0 otherwise.

  There is also a malevolent agent who is allowed to make payments to the agents,
  $\tilde{\gamma}(x, X, S)$, but the malevolent can also condition their payments on $S$.

  Agents report to maximize:

  $$V(x, X, S, l) = \max_{x \in \{0, 1\}} (p' + \frac{T}{K}) \gamma(x, X) + \tilde{\gamma}(x, X, S) - \mathcal{I}_{x \neq S} l$$

  Thus, given $\gamma$ and $\tilde{\gamma}$, we can compute an implied distribution over $X$ values.
  Let this distribution be called $G$.

  The malevolent would like to corrupt the system\footnote{Here, corrupt means to raise the
  probability of corruption above a threshold $\xi_m$} as cheaply as possible. The cost at which the
  malevolent can corrupt the system is given by

  \begin{align*}
    C^M(\gamma(x, X), S) &= \min_{\tilde{\gamma}(x, X, S)} \sum_{X} g(X) \sum_{l} f(l) \tilde{\gamma}(x_l, X, S) \\
    &\text{subject to} \\
    &1 - G(K/2) > \xi_m
  \end{align*}

  The oracle would like to report the truth at the minimum cost. It's problem is given by

  \begin{align*}
    V^O &= \min_{\gamma(x, X)} \sum_X g(X) \sum_l f(l) \gamma(x_l, X) \\
    &\text{subject to } \\
    &\sum_l \gamma(x_l, X) \leq K \quad \text{(Budget Constraint)} \\
    &C^M(S, \gamma(x, X)) \geq mN \quad \text{Malevolent Incompatible}
  \end{align*}

  The Malevolent Incompatible constraint is what ensures that the oracle chooses a payment scheme
  that cannot be corrupted because it ensures that $\text{CoC} > \text{PfC}$.


\section{Game Plan}

  The game plan is to solve for a discretized version of $\gamma(x, X)$ and determine the the
  corresponding $C^M(S, \gamma(x, X))$. With these in hand, we can likely pick a function that
  approximates $\gamma(x, X)$ ``well enough'' which mostly means that it doesn't require too
  much taxation on the trading system and that it satisfies the constraints of the oracle's problem.

  Once we have a solution to the static problem, we will move onto the dynamic problem. This will
  allow for $p'$ to be an endogenous object. The dynamic problem will also give the oracle more
  tools, such as positive reputation, to encourage truthful reporting. The malevolent agent does not
  have access to dynamic incentives because upon success the system shuts down. This should mean
  that the static results are still a solution to the truthful oracle problem (albeit an expensive
  one)...


\section{Progress}

  \subsection{Given an Oracle Policy}

    Imagine that we take as given an oracle policy. We will consider two potential policies:

    \begin{enumerate}
      \item No redistribution: $\gamma(x, X) = 1 \forall x, X$
      \item Complete redistribtion: $\gamma(x, X) = \begin{cases}
      \frac{K}{X} \quad \text{if } X < \frac{K}{2} \text{ and } x = 0 \\
      \frac{K}{K - X} \quad \text{if } X \geq \frac{K}{2} \text{ and } x = 1 \\
      0 \quad \text{else} \end{cases}$
    \end{enumerate}

    With these two policies in hand, we can determine what the optimal corruption policy of the
    malevolent agent is. Using this corruption policy, we can do ``back of the envelope'' type
    calculations that allow us to back out the required tax rate to prevent corruption.

    Using a discretized version of the model with the following parameters:

    \begin{itemize}
      \item Number of voters: $K=5$
      \item Lie penalty distribution: $l \sim N(1, 0.5)$
      \item Total tax revenue: $T = 0.05$
      \item Interest rate: $r = 0.025$
      \item Probability of attack success: $\xi_m = 0.65$
    \end{itemize}

    The cost of attacking with the given policies and parameters is

    \begin{itemize}
      \item No redistribution: $3.40$
      \item Full redistribution: $3.55$
    \end{itemize}

    Now consider letting the dollar amount of the profit from corruption ($mn$) be set to \$1,000
    for the complete redistribution policy. In the table below, we do the following thought
    experiments:

    \begin{enumerate}
      \item In $CoC = PfC + \varepsilon$, we set the cost of corruption (CoC) to the same value
      as the profit from corruption (PfC). We can then convert the total tax revenue into a dollar
      amount using the model-example exchange rate and use the identity $T = 2 \tau \text{PfC}$ to
      find the tax rate that achieves the PfC desired.
      \item In $\tau = 5\%$, we set the tax rate on the margin to $5\%$. We can then use the
      same identity as in last example but to find the total tax revenue. This gives us a
      model-example exchange rate which then tells us what the cost of corruption would be.
    \end{enumerate}

    \begin{center}
    \begin{tabular}{l|ccc}
      & Model & Example w $CoC = PfC + \varepsilon$ & Example with $\tau = 5\%$ \\
      \hline \\
      CoC & 3.55 & \$1,000 &  \$2,000 \\
      PfC & (?) & \$1,000 & \$1,000 \\
      T & 0.05 & \$14.05 & \$100 \\
      $\tau$ & (?) & 0.7\% & 5\% \\
    \end{tabular}
    \end{center}

  \subsection{Theorems}

    \begin{thm}
      $\gamma(x, Y) = \gamma(1 - x, K - Y)$
    \end{thm}

    \begin{thm}
      $\tilde{\gamma}(1 - S, X, S) = 0$
    \end{thm}

    \begin{thm}
      $\tilde{\gamma}(0, X, 1) = \tilde{\gamma}(1, K-X, 0)$
    \end{thm}

    \begin{thm}
      $\exists \bar{l}$ such that $\forall l > \bar{l}$ $x^*(l) = S$ and $\forall l < \bar{l}$ $x^*(l) = 1-S$
    \end{thm}

\section{Appendix A}

  To generate the assumption of $T = 2 \tau N$ and an amount of seizable margin of $mN$ we assume:

  \begin{enumerate}
    \item There are $N$ contracts held in the trading protocol.
    \item These contracts are symmetric in the sense that each counterparty holds $m$ margin.
    \item The malevolent is a counterparty on each of the outstanding contracts and if the wrong
    state is reported, the malevolent can extract each of their counterparty's margin.
    \item The counterparty of each contract is responsible for paying a tax, $\tau$, in order to
    be provided access to the oracle's information.
  \end{enumerate}

  Note that in some ways, this is a worst case analysis. The malevolent agent being a part of each
  contract is what drives the PfC to $mN$. In practice, the PfC would typically be lower than this.


\end{document}
