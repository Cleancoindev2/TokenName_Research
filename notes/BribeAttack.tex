\documentclass[12pt]{article}

  % Get right packages
  \usepackage{amsmath}
  \usepackage{amssymb}
  \usepackage{amsthm}
  \usepackage{float}
  \usepackage{fullpage} % Package to use full page
  \usepackage{graphicx}
  \usepackage{hyperref}
  \usepackage{parskip} % Package to tweak paragraph skipping

  % User commands
  \newtheorem{thm}{Theorem}
  \DeclareMathOperator*{\argmax}{argmax}

  % Title info
  \title{Bribery Attack Notes}
  \author{UMA Project}
  \date{\today}

\begin{document}

\maketitle
\clearpage
\newpage

\section{Overview}

\section{Basic bribery attack setting}

  One of the pieces of feedback that we have received is that the UMA Protocol is exposed to the
  simplest form of bribery. For the sake of clarity, we have tried to transform the arguments made
  by others into a formal game theory model.

  \subsection{Formal setting} \label{bba_fs}

    We consider the setting for the data verification machine (DVM) under which the system issues
    $\pi$ tokens to any individual who votes with the majority. The objective of the DVM is to
    provide accurate information by incentivizing people to vote ``correctly.'' We normalize the
    tokens and the rewards such that they are expressed in dollar terms. Note we assume that if the
    system becomes corrupted then the tokens are worth 0 in dollar terms.

    There is a continuum (of measure one) of agents who can either vote to corrupt or to not
    corrupt. There exists a 3rd party individual who would like to corrupt the system, they offer
    a contract which rewards any agent who votes to corrupt $\tilde{x}$ (in dollars).

    We can then determine the payoffs for the action taken by each individual:

    \begin{itemize}
      \item If the system is corrupted and the individual voted to corrupt, they receive
        $\tilde{x}$
      \item If the system is corrupted and the individual voted to not corrupt, they receive $0$
      \item If the system is not corrupted and the individual voted to not corrupt, they receive
        $1 + \tilde{x}$
      \item If the system is not corrupted and the individual voted to not corrupt, they receive
        $1 + \pi$
    \end{itemize}

  \subsection{Equilibrium description}

    We consider the space of all mixed strategy Nash equilibrium (MSNE). That is, given a strategy
    $p$, which denotes the probability with which each individual will vote to not corrupt, there
    must not be any profitable deviation on the individual level.

    An individual's payoff is given by

    \begin{align*}
      V &= p (\mathbb{I}_{\text{corrupt}} \cdot 0 + \mathbb{I}_{\text{not corrupt}} \cdot (1 + \pi)) + (1 - p) (\mathbb{I}_{\text{corrupt}} \cdot \tilde{x} + \mathbb{I}_{\text{not corrupt}} \cdot (1 + \tilde{x}))
    \end{align*}

    Suppose $\exists$ $p > 0.5$ such that $p$ is a MSNE then it must be that

    \begin{align*}
      p (1 + \pi) + (1 - p) (1 + \tilde{x}) > (1 + \tilde{x}) \\
      p ((1 + \pi) - (1 + \tilde{x})) > 0 \\
      (1 + \pi) > (1 + \tilde{x}) \\
      \pi > \tilde{x}
    \end{align*}

    However, if $\pi > \tilde{x}$ then the only $p$ that is a MSNE is $p = 1$ which implies
    all individuals vote to not corrupt.

    Now suppose $\exists$ $p < 0.5$ such that $p$ is a MSNE then it must be that

    \begin{align*}
      p (0) + (1 - p) \tilde{x} > 0 \\
      \tilde{x} > 0
    \end{align*}

    However, if $p < 0.5$ and $\tilde{x} > 0$ then the only MSNE is $p = 0$ which implies that all
    individuals vote to corrupt.

    This demonstrates how we might be exposed to certain forms of bribery --- In the case in which
    $\tilde{x} > \pi$ the only MSNE is $p = 0$ which implies certain corruption of the DVM.


\section{Proportional reward bribery attack setting}

  While the bribery model in the previous section is useful in highlighting where the system might
  be vulnerable, it is missing some important features of the actual system that help secure the
  system against such attacks. Namely, rewards are proportional to the number of voters in the
  majority.

  \subsection{Formal setting}

    We consider a similar setting to Section ~\ref{bba_fs}. There is a DVM that would like to
    reveal the truth by incentivizing people to vote ``correctly.'' It does this by offering a
    total amount of rewards $\pi$ split among all individuals who vote with the majority. We
    continue to normalize the token value and rewards to dollar amounts and work with the
    assumption that corruption results in the vote token being valued at 0.

    There is a continuum (of measure one) of agents who can either vote to corrupt or to not
    corrupt. There exists a third party individual who would like to corrupt the system, they
    offer a contract which rewards any agent who votes to corrupt with $\tilde{x}$ dollars.

    We can determine the payoffs for the action taken by each individual as a function of the
    fraction of individuals who vote to not corrupt $p$.

    \begin{itemize}
      \item If the system is corrupted and the individual voted to corrupt, they receive
        $\tilde{x}$
      \item If the system is corrupted and the individual voted to not corrupt, they receive $0$
      \item If the system is not corrupted and the individual voted to not corrupt, they receive
        $1 + \tilde{x}$
      \item If the system is not corrupted and the individual voted to not corrupt, they receive
        $1 + \frac{1}{p} \pi$
    \end{itemize}

  \subsection{Equilibrium description}

    We continue to use MSNE as our equilibrium concept. Rewards are now written as

    \begin{align*}
      V &= p (\mathbb{I}_{\text{corrupt}} \cdot 0 + \mathbb{I}_{\text{not corrupt}} \cdot (1 + \frac{1}{p} \pi)) + (1 - p) (\mathbb{I}_{\text{corrupt}} \cdot \tilde{x} + \mathbb{I}_{\text{not corrupt}} \cdot (1 + \tilde{x}))
    \end{align*}

    Suppose $\exists$ $p > 0.5$ such that $p$ is a MSNE then it must be that

    \begin{align*}
      p (1 + \frac{1}{p} \pi) + (1 - p) (1 + \tilde{x}) > 1 + \tilde{x} \\
      p( (1 + \frac{1}{p} \pi) - (1 + \tilde{x})) > 0 \\
      \pi > p \tilde{x}
    \end{align*}

    This is quite similar to the outcome of last time. However, it's important to note that in
    order to successfully convince the ``$0.50 + \varepsilon$'' person, it will require that
    $\tilde{x}$ is twice as high as in the previous case. This results in a multiplicity of
    equilibria in which $\tilde{x} > \pi$, but the system does not become corrupted.

    In the case of $p < 0.5$ we see much of the same math because people have already assumed that
    the system will be corrupted --- If people already know the system will be corrupted then it
    is a self-fulfilling prophecy.

    If we instead consider a trembling hand equilibrium, we introduce additional motive for
    individuals to That being said, there are additional equilibria with $p$
    close to 0.5 and they each require that $\tilde{x}$ be higher it becomes even more difficult
    to corrupt the DVM

\end{document}
