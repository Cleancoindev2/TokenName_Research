%!TEX root = ../TaxationPlan.tex

\textbf{Theory}

In this appendix section, we return to the problem of finding a minimum cost buyback policy.
However, we will no longer impose a non-negativity constraint. The modified mathematical program
becomes

\begin{align*}
  \min_{\{X_t\}} \; & E \left[ \sum_{t} \left(\frac{1}{1 + r} \right) X_t \right] \\
  &\text{subject to} \\
  \quad & 2 PfC_t \leq E \left[ \sum_{s=0}^{\infty} \left(\frac{1}{1 + r}\right)^s X_{t+s} \right] \quad (\lambda_t)
\end{align*}

We will continue to focus on Markov policies, $X^*(M^t) = X^*(M_t)$, so, as before, we know:

\begin{align*}
  2 \gamma M_t &\leq E \left[ \sum_{s=0}^{\infty} \left( \frac{1}{1+r} \right)^s X^*(M^s) \right] \\
  &\leq \sum_{s=0}^{\infty} \left( \frac{1}{1+r} \right)^s E \left[X^*(M_s) | M_t \right]
\end{align*}

In the discrete Markov chain case (where $P$ denotes the Markov transition matrix), then this
reduces to:

\begin{align*}
  2 \gamma M_t &= \sum_{s=0}^{\infty} \left( \frac{1}{1+r} \right)^s E \left[X^*(M_s) | M_t \right] \\
  2 \gamma \vec{M} &= \sum_{s=0}^{\infty} \left( \frac{1}{1+r} \right)^s P^s \vec{X} \\
  &\dots \\
  \vec{X} &= 2 \gamma \left(I - \frac{1}{1 + r} P \right) \vec{M}
\end{align*}

\textbf{Interpretation}

For certain parameterizations, it is optimal to have periods of negative buybacks ($X_t < 0$). We
discuss how this should be interpreted and propose a potential mechanism to implement negative
buybacks.

Consider the process that determines the number of tokens at a given time:

\begin{align*}
  S_{t+1} &= \pi (S_t - B_t) \\
  p_t S_{t+1} &= \pi (p_t S_t - X_t) \\
  X_t &= p_t \frac{\pi S_t - S_{t+1}}{\pi}
\end{align*}

Thus if $X_t < 0$ then $S_{t+1} > \pi S_t$

\begin{align*}
  S_{t+1} > pi S_t \\
  \pi (S_t - B_t) > \pi S_t \\
  \rightarrow B_t < 0
\end{align*}

The negative buybacks mean that we need to be increase the supply of tokens by more than the typical
inflation rate. One generic mechanism that would allow for positive and negative buybacks is simply
a period-by-period auction.

\begin{itemize}
  \item If $B_t > 0$ then the DVM performs an auction and allows token holders to name a price at
        which they would be willing to sell a certain number of their tokens. It then uses $X_t$
        dollars from the rainy day fund to purchase the tokens from these individuals.
  \item If $B_t < 0$ then the DVM performs an auction to sell off newly minted tokens. It allows
        individuals to mark a price at which they would be willing to purchase tokens and collects
        $X_t$ dollars for the rainy day fund by selling to the individuals who value the tokens
        the most.
\end{itemize}

An additional benefit of this process is that it helps ensure that tokens stay in the hands of those
who value them most which provides additional protections for the DVM.
