%!TEX root = ../TaxationPlan.tex

We use a stochastic version of the logistic growth process:

\begin{align*}
  M_{t+1} = M_{t} \left(1 + g \left(1 - \frac{M_{t}}{\bar{M}} \right)\right) + \sigma(M_{t}) \varepsilon_{t+1}
\end{align*}

\textbf{State values}

We create create a vector of evenly spaced points between some initial value, $M_0$, and the
deterministic steady state, $\bar{M}$. These will serve as the state values that margin can take.
The more points we place between these two numbers, the more accurate our approximation will become.

\textbf{Transition matrix}

To determine the probability of transitioning between two points $M_i$ and $M_j$, we take the
difference in value of the conditional cumulative distributions. Thus, the probability of
transitioning from $M_i$ to $M_j$ is given by

\begin{align*}
  p_{ij} = F\left(M_j - M_i \left(1 + g \left(1 - \frac{M_i}{\bar{M}}\right) \right) \right)
\end{align*}

where $F$ is a Normal distribution with mean 0 and standard deviation $\sigma(M_i)$.

Additionally, we add a small positive probability (inversely proportional to the size of the system)
of transitioning from any state to a "disaster" state which is absorbing\footnote{This actually
results in the guaranteed (eventual) demise of the system.}.

\textbf{Accuracy Evaluation}

In order to give a sense on how successful this approximation is, we report some moments from the
original process and moments generated by the approximation with different numbers of points between
$M_0$ and $\bar{M}$
