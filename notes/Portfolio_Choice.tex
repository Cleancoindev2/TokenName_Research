\documentclass[12pt]{article}

  % Get right packages
  \usepackage{amsmath}
  \usepackage{amssymb}
  \usepackage{amsthm}
  \usepackage{fullpage} % Package to use full page
  \usepackage{hyperref}
  \usepackage{parskip} % Package to tweak paragraph skipping

  % User commands
  \newtheorem{thm}{Theorem}
  \DeclareMathOperator*{\argmax}{argmax}

  % Title info
  \title{Portfolio Choice Problem}
  \author{UMA Project}
  \date{\today}

\begin{document}

\maketitle


\section{Baseline Heterogeneous Agent Model}

  Prior to discussing the details of the model, it is helpful to have a broad sense of what the
  inputs to the model will be:

  \begin{enumerate}
    \item There is an Oracle who is responsible for reporting some stochastic state, but is unable
          to determine this information on its own. In order to get the information from other
          individuals, the Oracle will issue voting tokens and reward individuals for reporting
          information, but must ensure that it is in people's interests to tell the truth.
    \item There are individuals who like consumption and participate in financial markets and can
          help (by reporting the truth), or hinder (by reporting the wrong state), the Oracle in its
          attempts to determine the stochastic state.
    \item Though not modeled directly, there is a financial market that depends on the information
          being provided by the Oracle. If the Oracle reports the wrong stochastic state then some
          of the collateral from this market can be stolen.
  \end{enumerate}

  These components give rise to various tradeoffs and we will look for policies that can sensibly
  ensure that individuals find it optimal to report the truth and guarantee that the collateral
  from the market is safe from being stolen.

  We now proceed to describe these components in more detail

  \subsection{Oracle}

    The Oracle issues total supply 1 of the vote token and desires to report some state
    $S_t \in \{0, 1\}$ correctly. The Oracle does not know the true value of $S_t$ but believes that
    $S_t = 1$ with probability $p_0 \geq 0.5$. The Oracle then uses votes by individuals as signals
    about the true state and will report $0$ if enough voters report $s = 0$\footnote{Enough can
    eventually be determined by treating the Oracle as a Bayesian decision maker who has a loss
    function over reporting a lie, postponing reporting, and reporting the truth.}.

    The Oracle's policy is to punish those who vote against its reported state by taking a fraction
    $\gamma$ of their vote tokens and by rewarding those who voted with majority by giving them the
    seized tokens. The majority receives $(1 - \gamma) \frac{\tilde{\pi}}{1 - \tilde{\pi}}$ where
    $\tilde{\pi}$ is the percent of individuals who vote against the majority.

    The Oracle is responsible for protecting $\mu_t$ margin in some system and generates revenue
    $\Pi_t = \tau_t(\mu_t) \mu_t$. $\mu_t$ follow some stochastic process. The Oracle pays the
    revenue out as dividends to vote token holders.

  \subsection{Individuals}

    Individuals are allowed access to two financial assets:

    \begin{enumerate}
      \item Risk-free bonds which pay return $r$
      \item Vote tokens issued by the Oracle which will pay an amount which depends on the
            individual's vote, $s_t$, and the Oracle's reported state, $\hat{S}_t$, according to
            $(1 + \gamma(s_t, \hat{S}_t)) (d_t + p_t)$ where $d_t$ is period $t$s dividend and
            $p_{t+1}$ is the value at which the token could be resold
    \end{enumerate}

    Voters who choose to vote dishonestly may successfully corrupt the Oracle. In order to
    successfully corrupt the Oracle, they must own enough votes to convince the Oracle to report
    the wrong state. If the dishonest voters successfully corrupt the Oracle then they are able to
    seize a fraction $\kappa$ of the margin in the system, but, afterwards, everyone will enter a
    new regime where they only have access to the risk-free bond, and, not the vote tokens.

    Individuals receive utility from consumption and maximize their consumption by choosing a
    portfolio of assets and to report honestly or dishonestly. Additionally, when voting,
    individuals who would like to report the truth forget to vote with probability $\eta$ and make a
    mistake with their vote with probability $\pi$. We assume that dishonest voters are more
    deliberate in the sense that they neither forget nor make mistakes in their votes.

    An agent with wealth $w_t$, with the system uncorrupted, and margin at $\mu_t$ faces the
    following problem:

    \begin{align}
      V(w_t, 0) &= \max_{b_{t+1}, x_{t+1}, s_{t}} u(c_t) + \beta E_x \left[ V(w_{t+1}, \text{Corrupt}_t) \right] \\
      w_{t+1} &= (1 + r) b_{t+1} + x_{t+1} (1 + \gamma(s_{t+1}, \hat{S}_{t+1})) (d_{t+1}(\mu_{t+1}) + p_{t+1}(\mu_{t+1}, \hat{S}_t)) \\
      c_t &= w_t - b_{t+1} - p_t x_{t+1}
    \end{align}

    and agent with wealth $w_t$ with a corrupted system faces

    \begin{align*}
      V(w_t, 1) &= \max_{b_{t+1}} u(c_t) + \beta V(w_{t+1}, 1) \\
      w_{t+1} &= (1 + r) b_{t+1} \\
      c_t &= w_t - b_{t+1}
    \end{align*}

    When computing expectations, agents believe that others will report the truth (besides for
    non-participation and trembling hand mistakes). Thus they will only attack if they can do so alone
    profitably.

  \subsection{Truthful Oracle}

    A \textit{Truthful Oracle} is an Oracle policy $(\gamma, \tau_t(\mu_t)$ such that
    $\hat{S}_t = 1$. This is done by ensuring that the truth-tellers value the token more than the
    liars do

    A \textit{Truthful Equilibrium} is

    \begin{enumerate}
      \item Value functions $V : \mathcal{R}^+ \times \mathcal{R}^+ \rightarrow \mathcal{R}$
      \item Portfolio decision functions $b^*, x^* : \mathcal{R}^+ \times \mathcal{R}^+ \rightarrow \mathcal{R}^{+}$
      \item Vote decision function $s^*: \mathcal{R}^+ \times \mathcal{R}^+ \rightarrow \{0, 1 \}$
      \item Price $p \in \mathcal{R}^+$
      \item Oracle redistribution policy: $\gamma \in (0, 1)$
    \end{enumerate}

    such that

    \begin{itemize}
      \item Value functions and policy functions are optimal given prices
      \item The Oracle policy produces a truthful oracle
      \item Token price, $p$, clears token market
    \end{itemize}

\end{document}
